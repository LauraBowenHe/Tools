% Multiple latex table. 
\documentclass[letterpaper]{article}
\usepackage{aaai}
\usepackage{times}
\usepackage{helvet}
\usepackage{courier}
\usepackage{multirow}
\usepackage{adjustbox}
\usepackage{pdfpages}
\usepackage{array}
\frenchspacing
\setlength{\pdfpagewidth}{8.5in}
\setlength{\pdfpageheight}{11in}
\pdfinfo{
/Title (Insert Your Title Here)
/Author (Put All Your Authors Here, Separated by Commas)}
\setcounter{secnumdepth}{0}  
 \begin{document}
% The file aaai.sty is the style file for AAAI Press 
% proceedings, working notes, and technical reports.
%
\title{Formatting Instructions \\for Authors Using \LaTeX{}}
\author{AAAI Press\\
Association for the Advancement of Artificial Intelligence\\
2275 East Bayshore Road, Suite 160\\
Palo Alto, California 94303\\
}
\maketitle


%跨双栏的表,宽度和页面一致,但字体不好控制
\begin{table*}[htbp]
\begin{adjustbox}{width=\textwidth}
\centering
  \begin{tabular}{|*{14}{r|}}
    \hline
\multicolumn{2}{|c|}{\multirow{2}*{}}
      &  \multicolumn{3}{|c|}{data} & \multicolumn{3}{|c|}{data} &  \multicolumn{3}{|c|}{data} & \multicolumn{3}{|c|}{data} \\\cline{3-14}
      \multicolumn{2}{|c|}{}  & acc   & acc & acc
      						 & acc & acc & acc 
                             & acc & acc & acc
                             & acc & acc & acc\\\hline
\multirow{2}*{rnn}
      & original & 24.37   & 16.18 & 19.62 & 8.98 & 31.98 & 18.71 & 622.76 & 266.16 & 31.98 & 18.71 & 622.76 & 266.16\\
      & input normalization & 24.37   & 16.18 & 19.62 & 8.98 & 31.98 & 18.71 & 622.76 & 266.16 & 31.98 & 18.71 & 622.76 & 266.16\\\hline
\multirow{3}*{lstm}
      & original & 7.156   & 25.607 & 189.793 & 70.336 & 286.354 & 86.888 & 6.446 & 18.434 & 31.98 & 18.71 & 622.76 & 266.16\\
      & input normalization & 5.960   & 7.043 & 21.980 & 10.078 & 2.780 & 13.793& 2.922 & 5.385 & 31.98 & 18.71 & 622.76 & 266.16\\\hline
\multirow{3}*{cnn}
      & original & 24.37   & 16.18 & 19.62 & 8.98 & 31.98 & 18.71 & 622.76 & 266.16 & 31.98 & 18.71 & 622.76 & 266.16\\
      & input normalization & 54.41   & 30.29 & 25.54 & 15.72 & 164.16 & 84.73 & 3226.60 & 1723.11 & 31.98 & 18.71 & 622.76 & 266.16\\\hline
  \end{tabular}
  \label{tab:data}
\end{adjustbox}
\caption{Experiment Results}
\end{table*}

% 跨双栏的表,可以控制列的宽度,可以控制列之间的线
\begin{table*}[htbp]
\normalsize
\centering
\begin{adjustbox}{width=160mm}
\centering
\begin{tabular}{p{20mm}<{\centering}|p{20mm}<{\centering}p{20mm}<{\centering}p{15mm}<{\centering}p{20mm}<{\centering}p{20mm}<{\centering}p{30mm}<{\centering}p{15mm}<{\centering}}
\hline
\ Data  & Metrics   & Origin. &RNN & LSTM
      						 & CNN & XGBoost & improv. \\\hline
\multirow{3}*{dataset1}
      & P@1 & 24.37   & 16.18 & 19.62 & 19.62 & 8.98 & 8.98\\
      & P@3 & 24.37   & 16.18 & 19.62 & 19.62 & 8.98 & 8.98\\
      & P@5 & 24.37   & 16.18 & 19.62 & 31.98 & 1 & 8.98\\\hline
\multirow{3}*{dataset2}
      & P@1 & 7.156   & 25.607 & 19.62 & 189.793 & 286.354 & 8.98\\
      & P@3 & 24.37   & 16.18 & 19.62 & 19.62 & 8.98 & 8.98\\
      & P@5 & 5.960   & 7.043 & 19.62 & 2.780 & 1 & 8.98\\\hline
\multirow{4}*{dataset3}
      & nDCG@1 & 24.37   & 16.18 & 19.62 & 19.62 & 8.98 & 8.98\\
      & nDCG@3 & 7.156   & 25.607 & 19.62 & 189.793 & 286.354 & 8.98\\
      & nDCG@5 & 24.37   & 16.18 & 19.62 & 19.62 & 8.98 & 8.98\\
      & nDCG@10 & 54.41   & 30.29 & 19.62 & 25.54 & 164.16 & 8.98\\\hline
\multirow{4}*{dataset4}
      & nDCG@1 & 7.156   & 25.607 & 19.62 & 189.793 & 286.354 & 8.98\\
      & nDCG@3 & 24.37   & 16.18 & 19.62 & 19.62 & 8.98 & 8.98\\
      & nDCG@5 & 24.37   & 16.18 & 19.62 & 19.62 & 8.98 & 8.98\\
      & nDCG@10 & 54.41   & 30.29 & 25.54 & 164.16 & 19.62 & 8.98\\\hline
  \end{tabular}
  \label{tab:data}
\end{adjustbox}
\caption{Experiment Results, compared with self}
\end{table*}

% 双栏页面的单栏表格
\begin{table}[htbp]
\tiny
\begin{adjustbox}{width=\columnwidth}
\centering
  \begin{tabular}{c|cccc}
    \hline
\   & Queries   & Items & Rel. & Feats. \\\hline
\multirow{1}*{Dataset1}
      & 1
 & 2   & 2 & 1 \\
\multirow{1}*{Dataset2}
      & 1  & 2 & 2 & 1\\
\multirow{1}*{Dataset3}
      & 1  & 2 & 5 & 1\\
\multirow{1}*{Dataset4}
      & 1 & 2  & 5 &  1\\\hline
  \end{tabular}
  \label{tab:data}
\end{adjustbox}
\caption{Dataset}
\end{table

% 插入跨栏的图
\begin{figure*}
\centering
\includegraphics[width=16cm]{figure-0mm.pdf} 
\caption{Model}
\end{figure*}
% visio画图的时候,页边距太大,可以用开发工具-页面-print properties-left,right,above, below = 0mm;然后设计-大小-适应绘图
% 保存的图片要尽量用矢量图,visio保存的pdf好像就是默认的矢量图。 

\end{document}
